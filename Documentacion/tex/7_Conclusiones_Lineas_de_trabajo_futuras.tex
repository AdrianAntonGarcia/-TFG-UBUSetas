\capitulo{7}{Conclusiones y Líneas de trabajo futuras}

%Todo proyecto debe incluir las conclusiones que se derivan de su desarrollo. Éstas pueden ser de diferente índole, dependiendo de la tipología del proyecto, pero normalmente van a estar presentes un conjunto de conclusiones relacionadas con los resultados del proyecto y un conjunto de conclusiones técnicas. 
%Además, resulta muy útil realizar un informe crítico indicando cómo se puede mejorar el proyecto, o cómo se puede continuar trabajando en la línea del proyecto realizado. 


\section{Conclusiones}

Respecto a los requisitos del proyecto, creo que se han cumplido ofreciendo un clasificador de especies que puede servir de ayuda en la práctica de la micología, de una manera rápida y accesible para los usuarios, apoyándose en las diferentes claves dicotómicas e información disponible de manera local.

A nivel personal, he adquirido nuevos conocimientos relacionados con la minería de datos, clasificación de imágenes, Web scraping, web semántica, planificación, uso de Latex y  el desarrollo en Android, entre otros, permitiéndome manejarme con mayor soltura en estos campos.

Respecto a las dificultades encontradas, algunas han surgido por intentar abarcar una gran cantidad de especies, lo que ha significado tener que encontrar un gran número de imágenes y de información de setas, además de dificultar la tarea de implementar una clave dicotómica que contuviese todos los géneros clasificados.

La escasez de imágenes de algunas especies se ha resuelto gracias a aplicar técnicas de data augmentation así como poder usar diferentes claves dicotómicas gracias a la aplicación de web scraping. El problema de automatizar la recolección de las claves dicotómicas es que no todas se presentaban de manera uniforme en la página web y este hecho ha dado problemas para poder recuperarlas, teniendo que suprimir algunas de estas claves disponibles. Aún así, considero que se han conseguido suficientes claves para las especies disponibles.

En general estoy satisfecho con el trabajo realizado y con las aplicaciones propuestas.

\section{Líneas de trabajo futuras}

A continuación se muestra un listado de aplicaciones y mejoras que se podrían incorporar en futuros proyectos:

\begin{itemize}
	\item Implementar un sistema de localización en el que el usuario pueda compartir donde ha encontrado una especie de seta en concreto.
	\item Implementar nuevas bases de datos que suministren información de las diferentes setas además de la DBpedia.
	\item Expandir el clasificador aumentando el número de especies o imágenes
	\item Implementar un sistema de aprendizaje en el que la aplicación pregunte al usuario sobre una seta, y este tenga que adivinar la especie.
	\item Generar un sistema en el que el usuario se pueda registrar y llevar un listado de las setas que ha ido clasificando a lo largo de su uso con la aplicación.
	\item Almacenar la información recopilada por los usuarios en un servidor para usarla en otras aplicaciones como podría ser, indicar en que zonas hay determinadas especies y en que momento se han encontrado.
	\item Generar un mercado virtual en el que los usuarios puedan vender y comprar diferentes tipos de setas.
\end{itemize}
