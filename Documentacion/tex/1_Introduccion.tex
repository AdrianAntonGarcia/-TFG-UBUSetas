\capitulo{1}{Introducción}

La tarea de clasificar una seta es altamente compleja ya que las diferencias entre algunas especies son mínimas y se necesitan de expertos que las analicen detalladamente hasta clasificar con seguridad la especie. Además dependiendo de la fase de crecimiento en la que se encuentre la propia seta y los factores a los que se encuentre expuesta, como puede ser la humedad del ambiente, pueden provocar que la apariencia de la seta cambie dificultando aún más esta tarea. 

La clasificación de una seta debe realizarse de forma segura y detallada ya que hay que asegurarse que la seta no sea tóxica ni provoque perjuicios a aquella persona que la consuma.

Ante estas dificultades nace la idea de crear una aplicación que nos ayude con la identificación de la especie o género a la que pertenece una seta. Debe quedar claro desde el principio que esta aplicación solo debe servir de guía y no pretende sustituir los conocimientos de un experto, simplemente servir de apoyo o de ayuda en la difícil tarea de clasificar la especie a la que pertenece una seta. 

Por estos motivos los resultados obtenidos por el clasificador deben tomarse sólo como una ayuda y usarlos para encaminar nuestra búsqueda de la especie correcta en el buen camino.

Para hacer uso de la aplicación, deberemos introducir una foto de la seta a clasificar, bien directamente sacando una foto desde la cámara o desde la galería de nuestro móvil. Con esta foto la aplicación nos mostrará las especies que el clasificador ha determinado como más probables. Si pulsamos sobre cada especie mostrada se nos mostrará información de esa especie, una breve descripción, la comestibilidad, así como diferentes imágenes que podremos comparar con la seta a clasificar. Además de mostrar esta información, la aplicación nos hará preguntas mediante claves dicotómicas según los resultados obtenidos para identificar correctamente la seta. Las preguntas consistirán en una breve descripción de características de la seta, el usuario deberá elegir la que más se adecue a la seta para poder llegar a la especie concreta.

La aplicación se ejecuta íntegramente en el móvil, es decir, no necesitaremos de una conexión a Internet para ejecutar el clasificador ni obtener información de la seta. Actualmente el clasificador es capaz de ejecutarse en el propio móvil sin necesidad de un servidor externo en un tiempo casi instantáneo permitiendo diferenciar entre 173 especies de setas diferentes. La aplicación cuenta con una pequeña base de datos que contiene información de cada especie, si lo deseamos podemos acceder por Internet al link proporcionado en cada especie a la página en la Wikipedia de la especie.